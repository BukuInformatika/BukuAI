\chapter{Pengenalan Kecerdasan Buatan dan Scikit-Learn}
\section{Kecerdasan Buatan}
\subsection{Definisi Kecerdasan Buatan}       
    Kecerdasan Buatan biasa disebut dengan istilah AI (Artificial Intelligence ). AI sendiri merupakan suatu cabang dalam bisnis sains komputer sains dimana mengkaji tntang bagaimana cara untuk menlengkapi sebuah komputer dengan kemampuan atau kepintaran layaknya atau mirip dengan yang dimiliki manusia. Sebagai contoh, sebagaimana komputer dapat berkomunikasi dengan pengguna baik menggunakan kata, suara maupun lain sebagainya. Dengan kemampuan ini, diharapkan komputer mampu mengambil keputusan sendiri untuk berbagai kasus yang ditemuinya kemudian itulah yang disebut dengan kecerdasan buatan. Kecerdasan buatan adalah kemampuan komputer digital atau robot yang dikendalikan konputer untuk melakukan tugas yang umumnya dikaitkan dengan sesuatu yang cerdas. Istilah ini sering diterapkan pada proyek pengembangan sistem yang diberkahi dengan karakteristik proses intelektual manusia, seperti kemampuan untuk berpikir, menemukan makna, menggeneralisasi, atau belajar dari pengalaman masa lalu.

    Kecerdasan Buatan adalah salah satu bidang studi yang berhubungan dengan pemanfaatan mesin untuk memecahkan persoalan yang rumit dengan cara lebih manusiawi dan lebih bisa di pahami oleh manusia. Kecerdasan buatan makin canggih dengan kemampuan komputer dalam memperbarui pengetahuannya dengan banyaknya testing dan perkembangan target analisa. Untuk kecerdasan buatan ada banyak contoh dan jenisnya. Salah satu contoh yang paling terkenal dari Artificial Intelligence ialah Google Assistant. Google Assistant digunakan untuk kemudahan user dalam menemukan berbagai hal maupun penyetingan langsung terhadap smartphone yang digunakan dan masih banyak lagi.

\subsection{Sejarah Kecerdasan Buatan}
    Artificial intelligence merupakan inovasi baru di bidang ilmu pengetahuan. Mulai terbentuk sejak adanya komputer modern dan kira-kira terjadi sekitaran tahun 1940 dan 1950. Ilmu pengetahuan komputer ini khusus ditujukan dalam perancangan otomatisasi tingkah laku cerdas dalam sistem kecerdasan komputer. Pada awal 50-an, studi tentang “mesin berpikir” memiliki berbagai nama seperti cybernetics, teori automata, dan pemrosesan innformasi. Pada tahun 1956, para ilmuan jenius seperti Alan Turing, Norbert, Wiener, Claude Shannon dan Warren McCullough telah bekerja secara independen dibidang cybernetics, matematika, algoritma dan teori jaringan. Namun, seprang ilmuan komputer dan kognitif John McCarthy adalah orang yang dating dengan ide untuk bergabung dengan upaya penelitian terpisah ini kedalam satu bidang yang akan mempelajari topic baru untuk imajinasi manusia yaitu kecerdasan buatan. Dia adalah orang yang menciptakan istilah tersebut dan kemudian mendirikan laboratorium Kecerdasan Buatan di MIT dan Stan ford.

    Pada tahun 1956, McCarthy yang sama mendirikan Konferensi Dartmouth di Hanover, New Hampshire. Peneliti terkemuka dalam teori kompleksitas, simulasi bahasa, hubungan antara keacakan dan pemikiran kreatif, jaringan saraf diundang. Tujuan dari bidang penelitian yang baru dibuat adalah untuk mengembangkan mesin yang dapat mensimulasikan setiap aspek kecerdasab. Itulah sebabnya Konferensi Dartmouth 1956 dianggap sebagai kelahiran Kecerdasan Buatan. Sejak saat itu, Kecerdasa Buatan telah hidup melalui decade kemuliaan dan cemoohan, yang dikenal luas sebagai musim panas dan musim dingin AI. Musim panasnya ditandai dengan optimism dan dana besar, sedangkan musim dinginnya dihadapkan dengan pemotongan dana, ketidakkpercayaan dan pesimisme.

\subsection{Perkembangan Kecerdasan Buatan}
    Teknologi Artificial Intelligence semakin ramai dibahas dalam berbagai diskusi teknologi di seluruh dunia.Menurut kebanyakan orang, pekerjaan seperti kasir, operator telepon, pengendara truk, dan lainnya sangat berpeluang besar untuk tergantikan oleh Artificial Intelligence. Mengapa terjadi hal demikian? dikarenakan memang bahwa AI lebih ungul dalam hal kinerja, fitur dan lain sebagainya. Namun, dalam beberapa aspek memang pekerja manusia masih unggul dibandingkan AI itu sendiri. Para generasi muda yang ada di dunia terutama di daerah Asia terlihat sudah memahami fungsi dan efek dari AI dalam kehidupan kita sehari-hari. Berdasarkan survei yang dilakukan oleh Microsoft, terdapat 39 persen responden yang mempertimbangkan untuk menggunakan mobil tanpa pengemudi dan 36 persen lainnya setuju bahwa robot masa depan dengan software untuk beroperasi mampu meningkatkan produktivitas. Dari survey tersebut kita sebagai pengguna AI harus lebih bijaksana dalam pengembangan dan penggunaan dari AI sehingga tanpa memberikan efek samping terhadap etos kerja dan keseharian kita sebagai pengguna dalam kehidupan sehari-hari.